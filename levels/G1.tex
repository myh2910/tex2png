%%p01
Un triángulo tiene $\underline{\phantom{3}}$ vértices.
%%a01
\begin{enum}
  * $4$
  * $6$
  * $3$
  * $8$
\end{enum}
%%r01
$3$
%%p02
Triángulo o también llamado...?
%%a02
\begin{enum}
  * Alaín
  * Trígono
  * Triangulitis
  * Cuadrilátero
\end{enum}
%%r02
Trígono
%%p03
En la figura, hallar ``$x$''.
\begin{figure}[h]
  \begin{tikzpicture}[thick]
    \def\r{2.5}
    \tkzDefPoint(-20:\r){A}
    \tkzDefPoint(180:\r){B}
    \tkzDefPoint(-98:\r){C}
    \tkzDefBarycentricPoint(A=-2,C=5) \tkzGetPoint{D}
    \def\a{6mm}
    \tkzFillAngles[size=\a,fill=blue,opacity=.2](B,A,C C,B,A B,C,D)
    \tkzMarkAngles[size=\a,mark=none](B,A,C C,B,A B,C,D)
    \tkzLabelAngle(B,A,C){$41\dg$}
    \tkzLabelAngle(C,B,A){$x$}
    \tkzLabelAngle(B,C,D){$80\dg$}
    \tkzDrawPolygon(A,B,C)
    \tkzDrawSegment[dashed](C,D)
    \tkzLabelPoints[right](A)
    \tkzLabelPoints[left](B)
    \tkzLabelPoints[below](C)
  \end{tikzpicture}
\end{figure}
%%a03
\begin{task}
  * $52\dg$
  * $39\dg$
  * $61\dg$
  * $121\dg$
\end{task}
%%r03
$39\dg$
%%p04
Calcula el valor de ``$\alpha$''.
  \begin{figure}[h]
    \begin{tikzpicture}[thick]
      \def\r{3}
      \tkzDefPoint(-165:\r){A}
      \tkzDefPoint(59:\r){B}
      \tkzDefPoint(-15:\r){C}
      \tkzDefPointOnLine[pos=1.2](A,C)
      \tkzGetPoint{D}
      \tkzFillAngles[size=5mm,fill=blue,opacity=.2](C,A,B A,B,C D,C,B)
      \tkzMarkAngles[size=5mm,mark=none](C,A,B A,B,C D,C,B)
      \tkzLabelAngles[pos=1](C,A,B){$37\dg$}
      \tkzLabelAngle[pos=.9](A,B,C){$\alpha$}
      \tkzLabelAngle[pos=1](D,C,B){$112\dg$}
      \tkzDrawPolySeg(D,A,B,C)
    \end{tikzpicture}
  \end{figure}
%%a04
\begin{task}
  * $78\dg$
  * $75\dg$
  * $73\dg$
  * $74\dg$
\end{task}
%%r04
$75\dg$
%%p05.sp
\begin{mini}
  El perímetro del triángulo es de $\ce{20cm}$. ¿Cuánto mide el lado $AC$?
  \begin{center}
    \begin{tikzpicture}[thick]
      \def\r{3}
      \def\a{16.6}
      \tkzDefPoint(\a+146.8:\r){A}
      \tkzDefPoint(\a:\r){B}
      \tkzDefPoint(\a+50.42:\r){C}
      \draw (A) node[below left] {$A$} -- node[below] {$x+5$} (B) node[below right] {$B$} -- node[above right] {$x$} (C) node[above] {$C$} -- node[above left] {$x+3$} (A);
    \end{tikzpicture}
  \end{center}
\end{mini}
%%a05
\begin{task}
  * $\ce{5cm}$
  * $\ce{2cm}$
  * $\ce{7cm}$
  * $\ce{6cm}$
\end{task}
%%r05
$\ce{7cm}$
%%p06
En la figura, hallar ``$x$''.
\begin{figure}[h]
  \begin{tikzpicture}[thick]
    \def\r{2.5}
    \tkzDefPoint(-150:\r){A}
    \tkzDefPoint(-30:\r){B}
    \tkzDefPoint(60:\r){C}
    \def\a{6mm}
    \tkzFillAngles[size=\a,fill=blue,opacity=.2](B,A,C C,B,A A,C,B)
    \tkzMarkAngles[size=\a,mark=none](B,A,C C,B,A A,C,B)
    \tkzLabelAngle(B,A,C){$3x$}
    \tkzLabelAngle(C,B,A){$5x$}
    \tkzLabelAngle(A,C,B){$4x$}
    \tkzDrawPolygon(A,B,C)
    \tkzLabelPoints[below left](A)
    \tkzLabelPoints[below right](B)
    \tkzLabelPoints[above](C)
  \end{tikzpicture}
\end{figure}
%%a06
\begin{task}
  * $15\dg$
  * $18\dg$
  * $20\dg$
  * $10\dg$
\end{task}
%%r06
$15\dg$
%%p07
Si $ABCD$ es un paralelogramo, calcular ``$x$''.
\begin{figure}[h]
  \begin{tikzpicture}[thick]
    \def\x{4}
    \def\y{3.8}
    \tkzDefPoints{0/0/A,\x/0/D}
    \tkzDefPoint(144:\y){B}
    \tkzDefParallelogram(D,A,B) \tkzGetPoint{C}
    \tkzFillAngles[size=5mm,fill=blue,opacity=.2](B,C,D C,D,A)
    \tkzMarkAngles[size=5mm,mark=none](B,C,D C,D,A)
    \tkzLabelAngle[pos=.8](B,C,D){$4x$}
    \tkzLabelAngle(C,D,A){$x$}
    \tkzDrawPolygon(A,B,C,D)
    \tkzLabelPoints[below left](A)
    \tkzLabelPoints[above left](B)
    \tkzLabelPoints[above right](C)
    \tkzLabelPoints[below right](D)
  \end{tikzpicture}
\end{figure}
%%a07
\begin{task}
  * $24\dg$
  * $36\dg$
  * $20\dg$
  * $72\dg$
\end{task}
%%r07
$36\dg$
%%p08
En el romboide $ABCD$, calcular ``$x$''.
\begin{figure}[h]
  \begin{tikzpicture}[thick]
    \tkzDefPoints{0/0/A,4/0/D}
    \tkzDefPoint(75:3){B}
    \tkzDefParallelogram(B,A,D) \tkzGetPoint{C}
    \def\r{6mm}
    \tkzFillAngles[size=\r,fill=blue,opacity=.2](A,B,C B,C,D)
    \tkzMarkAngles[size=\r,mark=none](A,B,C B,C,D)
    \tkzLabelAngle(A,B,C){$7x$}
    \tkzLabelAngle(B,C,D){$5x$}
    \tkzDrawPolygon(A,B,C,D)
    \tkzLabelPoints[below left](A)
    \tkzLabelPoints[above left](B)
    \tkzLabelPoints[above right](C)
    \tkzLabelPoints[below right](D)
  \end{tikzpicture}
\end{figure}
%%a08
\begin{task}
  * $12\dg$
  * $15\dg$
  * $10\dg$
  * $8\dg$
\end{task}
%%r08
$15\dg$
%%p09.sp
\begin{mini}
  ¿Cuántas diagonales se pueden trazar en un hexágono?
\end{mini}
%%a09
\begin{enum}
  * $6$
  * $9$
  * $12$
  * $7$
\end{enum}
%%r09
$9$
%%p10.sp
\begin{mini}
  ¿Cuántas diagonales se pueden trazar en total en un polígono de $28$ lados?
\end{mini}
%%a10
\begin{enum}
  * $350$
  * $250$
  * $170$
  * $280$
\end{enum}
%%r10
$350$
%%p11.sp
\begin{mini}
  Calcula la suma de las medidas de los ángulos internos de un decágono.
\end{mini}
%%a11
\begin{enum}
  * $1080\dg$
  * $1260\dg$
  * $1440\dg$
  * $900\dg$
\end{enum}
%%r11
$1440\dg$
%%p12.sp
\begin{mini}
  ¿Cuánto mide el ángulo externo de un icoságono regular?
\end{mini}
%%a12
\begin{enum}
  * $12\dg$
  * $16\dg$
  * $10\dg$
  * $18\dg$
\end{enum}
%%r12
$18\dg$
%%p13.sp
\begin{mini}
  ¿Cuánto suman los ángulos externos e internos de un icoságono?
\end{mini}
%%a13
\begin{enum}
  * $2800\dg$
  * $8200\dg$
  * $3600\dg$
  * $4000\dg$
\end{enum}
%%r13
$3600\dg$
%%p14.sp
\begin{mini}
  Si se triplica el número de lados de un polígono, la suma de sus ángulos internos queda quintuplicada. ¿Cómo se llama dicho polígono?
\end{mini}
%%a14
\begin{enum}
  * Triángulo
  * Alaín
  * Cuadrilátero
  * Octógono
\end{enum}
%%r14
Cuadrilátero
%%p15.sp
\begin{mini}
  Hallar la suma de los números de diagonales de un nonágono y de un endecágono.
\end{mini}
%%a15
\begin{task}
  * $27$
  * $44$
  * $71$
  * $69$
\end{task}
%%r15
$71$
