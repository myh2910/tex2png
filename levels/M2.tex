%%p01
$\dfrac23-\dfrac34\cdot\dfrac23+\dfrac56$
%%a01
\begin{task}
  * $1$
  * $-1$
  * $2$
  * $\dfrac12$
\end{task}
%%r01
$1$
%%p02
$\left(\dfrac12+\dfrac56\right)\div\dfrac43-\dfrac15$
%%a02
\begin{task}
  * $\dfrac46$
  * $-\dfrac46$
  * $\dfrac45$
  * $-\dfrac45$
\end{task}
%%r02
$\dfrac45$
%%p03
$\dfrac{\dfrac56+\dfrac12}{\dfrac23-\dfrac59}$
%%a03
\begin{task}
  * $\dfrac{1}{12}$
  * $12$
  * $-\dfrac{1}{12}$
  * $-12$
\end{task}
%%r03
$12$
%%p04
$\dfrac{\left(\dfrac23\div\dfrac32\right)}{\left(\dfrac53\cdot\dfrac12\right)}$
%%a04
\begin{task}
  * $\dfrac{8}{15}$
  * $\dfrac{4}{12}$
  * $\dfrac{15}{2}$
  * $-\dfrac{2}{15}$
\end{task}
%%r04
$\dfrac{8}{15}$
%%p05
$\dfrac{\dfrac13+\dfrac14}{3\dfrac12-2}$
%%a05
\begin{task}
  * $\dfrac{14}{36}$
  * $\dfrac{36}{14}$
  * $\dfrac{28}{14}$
  * $\dfrac{24}{36}$
\end{task}
%%r05
$\dfrac{14}{36}$
%%p06
$\left[\dfrac65\div\dfrac{9}{10}-\left(2-\dfrac{7}{12}\right)\right]+\dfrac{7}{24}$
%%a06
\begin{task}
  * $\dfrac{5}{34}$
  * $-\dfrac{5}{23}$
  * $\dfrac{5}{24}$
  * $\dfrac{5}{32}$
\end{task}
%%r06
$\dfrac{5}{24}$
%%p07.sp
\begin{mini}[.6]
  Víctor, Daniel, Beto son militares con $3$ rangos distintos; soldado, cabo y
  mayor aunque no necesariamente en ese orden. Si:
  \begin{itemize}
    \ii Beto es el soldado
    \ii Daniel no es el cabo
  \end{itemize}
  ¿Cómo se llama el mayor?
\end{mini}
%%a07
\begin{mini}[.7]
  \begin{enum*}
    * Víctor
    * Daniel
    * Beto
    * F.D.
  \end{enum*}
\end{mini}
%%r07
Daniel
%%p08.sp
\begin{mini}
  Karina recorre los $\dfrac23$ de un camino en bicicleta y $\dfrac19$ a pie.
  ¿Qué parte del camino recorrió en total?
\end{mini}
%%a08
\begin{task}
  * $\dfrac57$
  * $\dfrac79$
  * $\dfrac23$
  * $\dfrac37$
\end{task}
%%r08
$\dfrac79$
%%p09.sp
\begin{mini}
  La razón geométrica de dos números vale $\dfrac47$ y su razón aritmética es
  $45$. Determina el menor de los números.
\end{mini}
%%a09
\begin{task}
  * $50$
  * $45$
  * $60$
  * $52$
\end{task}
%%r09
$60$
%%p10.sp
\begin{mini}
  La suma de dos números es $144$ y su razón geométrica vale $\dfrac27$. ¿Cuáles
  son los dichos números?
\end{mini}
%%a10
\begin{enum}
  * $100$ y $44$
  * $81$ y $63$
  * $32$ y $112$
  * $90$ y $36$
\end{enum}
%%r10
$32$ y $112$
%%p11.sp
\begin{mini}[.7]
  $6$ amigas se sientan alrededor de una mesa circular. María, que está sentada
  a la derecha de Paola, se encuentra frente a Noemí. Paola está frente a la que
  está junto y a la derecha de Sara, que está frente a Raquel. ¿Quién está junto
  y a la derecha de Carmen?
\end{mini}
%%a11
\begin{mini}[.7]
  \begin{enum*}
    * Raquel
    * Paola
    * María
    * Noemí
  \end{enum*}
\end{mini}
%%r11
Noemí
%%p12.sp
\begin{mini}
  Si $a,b$ son números positivos tales que $\dfrac ab=\dfrac{7}{11}$ y además
  $a\cdot b=308$, calcular $b-a$.
\end{mini}
%%a12
\begin{task}
  * $2$
  * $3$
  * $7$
  * $8$
\end{task}
%%r12
$8$
%%p13.sp
\begin{mini}
  Tres números son entre sí como $4$, $7$ y $11$; y la suma del menor con el
  mayor de dichos números es $105$. Determinar el menor de estos números.
\end{mini}
%%a13
\begin{task}
  * $49$
  * $14$
  * $24$
  * $28$
\end{task}
%%r13
$28$
%%p14.sp
\begin{mini}
  Las edades de Juan y Rocio están en relación de $5$ a $9$ y la suma de ellas
  es $84$. ¿Qué edad tiene Juan?
\end{mini}
%%a14
\begin{task}
  * $20$
  * $30$
  * $40$
  * $45$
\end{task}
%%r14
$30$
%%p15.sp
\begin{mini}
  Tres números están en la misma relación que $5$, $9$ y $13$. Si la suma de
  ellos es $216$, indica el mayor de ellos.
\end{mini}
%%a15
\begin{task}
  * $102$
  * $88$
  * $104$
  * $96$
\end{task}
%%r15
$104$
