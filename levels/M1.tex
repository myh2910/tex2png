%%p01
$\dfrac{17}{36}-\dfrac{11}{24}$
%%a01
\begin{enum}
	* $\dfrac{1}{72}$
	* $-\dfrac{1}{72}$
	* $\dfrac{1}{144}$
	* $-\dfrac{1}{144}$
\end{enum}
%%r01
$\dfrac{1}{72}$
%%p02
$\dfrac{9}{10}+\dfrac{5}{12}-\dfrac{11}{15}$
%%a02
\begin{enum}
	* $\dfrac{9}{12}$
	* $-\dfrac{7}{12}$
	* $\dfrac{7}{12}$
	* $\dfrac{8}{12}$
\end{enum}
%%r02
$\dfrac{7}{12}$
%%p03
$\dfrac{2}{5}\div\dfrac{1}{4}\div\dfrac{3}{4}$
%%a03
\begin{enum}
	* $\dfrac{42}{32}$
	* $\dfrac{15}{-33}$
	* $\dfrac{32}{15}$
	* $\dfrac{12}{15}$
\end{enum}
%%r03
$\dfrac{32}{15}$
%%p04
$3\dfrac{2}{3}\times 2\dfrac{1}{7}$
%%a04
\begin{enum}
	* $\dfrac{7}{55}$
	* $-\dfrac{55}{7}$
	* $\dfrac{55}{8}$
	* $\dfrac{55}{7}$
\end{enum}
%%r04
$\dfrac{55}{7}$
%%p05
$\dfrac{3}{4}+\dfrac{7}{2}-\dfrac{3}{8}$
%%a05
\begin{enum}
	* $3\dfrac{7}{8}$
	* $7\dfrac{6}{5}$
	* $6\dfrac{3}{2}$
	* $4\dfrac{9}{8}$
\end{enum}
%%r05
$3\dfrac{7}{8}$
%%p06
$\dfrac{8}{9}\div\dfrac{1}{2}$
%%a06
\begin{enum}
	* $\dfrac{2}{4}$
	* $\dfrac{7}{12}$
	* $\dfrac{8}{13}$
	* $\dfrac{16}{9}$
\end{enum}
%%r06
$\dfrac{16}{9}$
%%p07
$8\dfrac{1}{2}\times\dfrac{2}{5}$
%%a07
\begin{enum}
	* $\dfrac{7}{6}$
	* $\dfrac{17}{5}$
	* $\dfrac{4}{6}$
	* $\dfrac{5}{11}$
\end{enum}
%%r07
$\dfrac{17}{5}$
%%p08
\begin{tabular}{c}
	Escribe la fracci\'on que se representa el siguiente gr\'afico: \vspace{5pt} \\
	\begin{tikzpicture}[fill=blue!20,draw=blue!60,thick]
		\matrix[matrix of nodes, nodes={draw,minimum size=1cm}, nodes in empty cells,column sep=-\pgflinewidth,row sep=-\pgflinewidth]{
			|[fill]| & |[fill]| &          & |[fill]| & |[fill]| \\
			         & |[fill]| & |[fill]| & |[fill]| &          \\
			         & |[fill]| & |[fill]| & |[fill]| &          \\
			|[fill]| & |[fill]| &          & |[fill]| & |[fill]| \\
		};
	\end{tikzpicture}
\end{tabular}
%%a08
\begin{task}
	* $\dfrac{15}{20}$
	* $\dfrac{14}{20}$
	* $\dfrac{19}{14}$
	* $\dfrac{25}{13}$
\end{task}
%%r08
$\dfrac{14}{20}$
%%p09
$\dfrac{2^2+3^2+4^2}{3-1}$
%%a09
\begin{enum}
	* $\dfrac{29}{3}$
	* $\dfrac{2}{29}$
	* $-\dfrac{29}{2}$
	* $\dfrac{29}{2}$
\end{enum}
%%r09
$\dfrac{29}{2}$
%%p10.sp
\begin{mini}[.8]
	Luz, Mar\'ia, Nancy, Ofelia, Pamela y Rita se sientan alrededor de una mesa circular con seis asientos distribuidos sim\'etricamente. Si se sabe que:
	\begin{itemize}
		\ii Ofelia no se sienta junto a Mar\'ia.
		\ii Pamela no se sienta junto a Nancy.
		\ii Luz se sienta junto y a la derecha de Mar\'ia y frente a Nancy.
	\end{itemize}
	¿D\'onde se sienta Rita?
\end{mini}
%%a10
\begin{mini}[.9]
	\begin{enum}(2)
		* Junto y entre Nancy y Pamela
		* Frente a Nancy
		* Junto y entre Mar\'ia y Nancy
		* Frente a Mar\'ia
	\end{enum}
\end{mini}
%%r10
\begin{tabular}{c}
	Junto y entre \\
	Mar\'ia y Nancy
\end{tabular}
%%p11
¿Qu\'e fracci\'on es equivalente a $\dfrac{6}{8}$?
%%a11
\begin{task}
	* $\dfrac{3}{4}$
	* $\dfrac{2}{4}$
	* $\dfrac{5}{4}$
	* $\dfrac{1}{4}$
\end{task}
%%r11
$\dfrac{3}{4}$
%%p12.sp
\begin{mini}[.8]
	Alrededor de una mesa circular est\'an sentadas $6$ amigas distribuidas sim\'etricamente. Si se sabe que:
	\begin{itemize}
		\ii Karen se ubica junto a Rosa pero no junto a Mar\'ia.
		\ii Ana se sienta frente a la persona que est\'a junto y a la izquierda de Rosa.
		\ii Mar\'ia est\'a a dos lugares de Ana.
		\ii In\'es se ubica a dos lugares a la derecha de Dora.
	\end{itemize}
	¿Qui\'en se encuentra frente a Ana?
\end{mini}
%%a12
\begin{task}
	* Dora
	* Rosa
	* Mar\'ia
	* In\'es
\end{task}
%%r12
In\'es
%%p13.sp
\begin{mini}[.8]
	Tres amigas: Carmen, F\'atima y Milagros, comentan sobre el color de polo que llevan puesto.
	\begin{itemize}
		\ii Carmen dice: ``Mi polo no es rojo ni azul como los de ustedes''.
		\ii Milagros dice: ``Me gustar\'ia tener un polo verde como el tuyo''.
		\ii F\'atima dice: ``Me gusta mi polo rojo''.
	\end{itemize}
	¿Qu\'e color de polo tiene Carmen?
\end{mini}
%%a13
\begin{mini}[.7]
	\begin{enum*}
		* Rojo
		* Azul
		* Verde
		* Rosado
	\end{enum*}
\end{mini}
%%r13
Verde
%%p14.sp
\begin{mini}
	Karina recorre los $\dfrac{2}{3}$ de un camino en bicicleta y $\dfrac{1}{9}$ a pie. ¿Qu\'e parte del camino recorri\'o en total?
\end{mini}
%%a14
\begin{task}
	* $\dfrac{5}{7}$
	* $\dfrac{7}{9}$
	* $\dfrac{2}{3}$
	* $\dfrac{3}{7}$
\end{task}
%%r14
$\dfrac{7}{9}$
%%p15.sp
\begin{mini}[.6]
	V\'ictor, Daniel, Beto son militares con $3$ rangos distintos; soldado, cabo y mayor aunque no necesariamente en ese orden. Si:
	\begin{itemize}
		\ii Beto es el soldado
		\ii Daniel no es el cabo
	\end{itemize}
	¿C\'omo se llama el mayor?
\end{mini}
%%a15
\begin{mini}[.7]
	\begin{enum*}
		* V\'ictor
		* Daniel
		* Beto
		* F.D.
	\end{enum*}
\end{mini}
%%r15
Daniel
