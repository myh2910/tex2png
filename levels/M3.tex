%%p01.sp
\begin{mini}
  Susana es una doctora que trabaja en un hospital de Huancayo. Ella utiliza $2\dfrac{1}{2}$ litros de suero para un paciente y $1\dfrac{2}{6}$ litros para otro. Calcula cuántos litros de suero utiliza.
\end{mini}
%%a01
\begin{task}
  * $\dfrac{23}{7}$
  * $\dfrac{23}{6}$
  * $-\dfrac{23}{6}$
  * $\dfrac{6}{23}$
\end{task}
%%r01
$\dfrac{23}{6}$
%%p02.sp
\begin{mini}[.8]
  Tres amigas: Carmen, Fátima y Milagros, comentan sobre el color de polo que llevan puesto.
  \begin{itemize}
    \ii Carmen dice: ``Mi polo no es rojo ni azul como los de ustedes''.
    \ii Milagros dice: ``Me gustaría tener un polo verde como el tuyo''.
    \ii Fátima dice: ``Me gusta mi polo rojo''.
  \end{itemize}
  ¿Qué color de polo tiene Carmen?
\end{mini}
%%a02
\begin{mini}[.7]
  \begin{enum*}
    * Rojo
    * Azul
    * Verde
    * Rosado
  \end{enum*}
\end{mini}
%%r02
Verde
%%p03.sp
\begin{mini}[.7]
  Seis amigos viven en un edificio, cada uno en un piso diferente. Carlos vive más abajo que Bica, pero más arriba que David. Franco vive $3$ pisos más abajo que Carlos. Andrés vive $2$ pisos más arriba que Carlos y a $4$ pisos de Enzo. ¿El tercer piso lo ocupa?
\end{mini}
%%a03
\begin{task}
  * Bica
  * David
  * Franco
  * Carlos
\end{task}
%%r03
David
%%p04.sp
\begin{mini}[.8]
  Seis amigos se sientan alrededor de una mesa circular en seis asientos simétricamente distribuidos. Se conoce lo siguiente:
  \begin{itemize}
    \ii Ernesto está al frente de Carla.
    \ii Dina está al frente de Flor, quien no está junto a Alonso.
    \ii Carla está junto y a la derecha de Alonso.
  \end{itemize}
  ¿Quíen está junto y a la izquierda de Beto?
\end{mini}
%%a04
\begin{mini}[.7]
  \begin{enum*}
    * Carla
    * Flor
    * Dina
    * Ernesto
  \end{enum*}
\end{mini}
%%r04
Flor
%%p05.sp
\begin{mini}[.8]
  Milton, Néstor, Paúl y Renzo se sentaron a beber alrededor de una mesa circular, en asientos simétricamente dispuestos. El que se sentó a la izquierda de Néstor bebió agua; Milton estaba frente al que bebía vino. Quien se sentaba a la derecha de Renzo, bebía gaseosa, el que bebía café y el que bebía gaseosa estaban frente a frente. ¿Qué bebía Paúl y quién tomaba vino respectivamente?
\end{mini}
%%a05
\begin{enum}
  * Gaseosa - Milton
  * Vino - Paúl
  * Gaseosa - Renzo
  * Agua - Milton
\end{enum}
%%r05
Gaseosa - Renzo
%%p06.sp
\begin{mini}[.9]
  En una reunión se encuentran $6$ amigos: Carlos, Néstor, Juana, Lizandro, María y Victoria, quienes se sientan alternadamente varones y mujeres en seis sillas igualmente espaciadas alrededor de una mesa circular. Se sabe lo siguiente:
  \begin{itemize}
    \ii Néstor se sienta a la derecha y junto a Victoria.
    \ii Carlos se sienta frente a Victoria.
  \end{itemize}
  ¿Cuáles de las siguientes afirmaciones son correctas?
  \begin{enumerate}[label=\Roman*)]
    \ii Néstor se sienta junto a María.
    \ii Lizandro se sienta junto a Victoria.
    \ii María no se sienta frente a Carlos.
  \end{enumerate}
\end{mini}
%%a06
\begin{task}
  * I y II
  * II y III
  * I y III
  * Todas
\end{task}
%%r06
II y III
%%p07.sp
\begin{mini}[.7]
  Carlos, Dante, Toño, Erick, Beto y Flavio se ubican en $6$ asientos contiguos en una hilera de un teatro. Toño está junto y a la izquierda de Beto, Carlos a la derecha de Toño entre Flavio y Dante, Dante está junto y a la izquierda de Erick. ¿Quién ocupa el tercer asiento si los contamos de izquierda a derecha?
\end{mini}
%%a07
\begin{mini}[.7]
  \begin{enum*}
    * Carlos
    * Erick
    * Dante
    * Flavio
  \end{enum*}
\end{mini}
%%r07
Flavio
%%p08.sp
\begin{mini}[.8]
  $3$ amigos estudiaron en la universidad; uno Física, otro Agronomía y otro Ingeniería. Cada uno de ellos tiene un hijo que cuando ingrese a la universidad, decidirá no tomar la carrera de su padre sino estudiar la carrera de uno de los amigos de su padre. Sabiendo que el Ingeniero se llama Luis y que el hijo de Juan quiere ser Agrónomo; ¿qué profesión tiene Juan y a qué quiere dedicarse el hijo de Rogelio?
\end{mini}
%%a08
\begin{enum}
  * Juan es Agrónomo y el hijo de Rogelio quiere ser Físico
  * Juan es Físico y el hijo de Rogelio quiere ser Ingeniero
  * Juan es Agrónomo y el hijo de Rogelio quiere ser Agrónomo
  * Juan es Físico y el hijo de Rogelio quiere ser Físico
\end{enum}
%%r08
\begin{tabular}{c}
  Juan es Físico y \\
  el hijo de Rogelio \\
  quiere ser Ingeniero
\end{tabular}
%%p09.sp
\begin{mini}[.8]
  Alrededor de una mesa circular están sentadas $6$ amigas distribuidas simétricamente. Si se sabe que:
  \begin{itemize}
    \ii Karen se ubica junto a Rosa pero no junto a María.
    \ii Ana se sienta frente a la persona que está junto y a la izquierda de Rosa.
    \ii María está a dos lugares de Ana.
    \ii Inés se ubica a dos lugares a la derecha de Dora.
  \end{itemize}
  ¿Quién se encuentra frente a Ana?
\end{mini}
%%a09
\begin{task}
  * Dora
  * Rosa
  * María
  * Inés
\end{task}
%%r09
Inés
%%p10.sp
\begin{mini}[.8]
  En la playa cuatro niñas: Anita, Luisita, Carmencita y Dianita forman una circunferencia al tomarse de las manos. Si:
  \begin{itemize}
    \ii La niña de ropa de baño verde está a la izquierda de Carmencita.
    \ii Luisita está al frente de la niña de ropa de baño rojo.
    \ii La niña a la derecha de Anita tiene ropa de baño de color rosado, y ésta se encuentra frente a la de ropa de baño amarilla.
  \end{itemize}
  ¿De qué color es la ropa de baño de Dianita?
\end{mini}
%%a10
\begin{mini}[.7]
  \begin{enum*}
    * Verde
    * Rojo
    * Rosado
    * Amarillo
  \end{enum*}
\end{mini}
%%r10
Rosado
%%p11.sp
\begin{mini}[.8]
  En un comedor de estudiantes, $8$ comensales se sientan en una mesa circular guardando distancias proporcionales. Las $8$ personas son estudiantes de diversas especialidades.
  \begin{itemize}
    \ii El de Ingeniería está frente al de Educación y entre (a los costados) los de Economía y Farmacia.
    \ii El de Periodismo está a la izquierda de el de Educación y frente al de Economía.
    \ii Frente al de Farmacia está el de Derecho; este a su vez está a la siniestra de el de Arquitectura.
  \end{itemize}
  ¿Cuál de ellos está entre los estudiantes de Biología y Educación?
\end{mini}
%%a11
\begin{mini}[.9]
  \begin{enum*}
    * Periodismo
    * Farmacia
    * Ingeniería
    * Economía
  \end{enum*}
\end{mini}
%%r11
Periodismo
%%p12.sp
\begin{mini}[.8]
  Fernanda, Silvia y Ana llegan a una fiesta a las $\ce{8pm}$, $\ce{9pm}$ y $\ce{10pm}$ aunque no necesariamente en ese orden y llevan puestos vestidos de color rojo, negro y turquesa no necesariamente en ese orden. Se tiene la siguiente información:
  \begin{itemize}
    \ii La que llegó a las $\ce{9pm}$, lleva vestido de color rojo.
    \ii Silvia que llegó última le comenta a la que lleva vestido negro, que ese color le queda espectacular.
  \end{itemize}
  Si Fernanda llegó antes que todas, ¿quién lleva el vestido negro y a qué hora llegó?
\end{mini}
%%a12
\begin{mini}
  \begin{enum}(2)
    * Ana - $\ce{8pm}$
    * Fernanda - $\ce{10pm}$
    * Silvia - $\ce{10pm}$
    * Fernanda - $\ce{8pm}$
  \end{enum}
\end{mini}
%%r12
Fernanda - $\ce{8pm}$
%%p13.sp
\begin{mini}[.8]
  Luz, María, Nancy, Ofelia, Pamela y Rita se sientan alrededor de una mesa circular con seis asientos distribuidos simétricamente. Si se sabe que:
  \begin{itemize}
    \ii Ofelia no se sienta junto a María.
    \ii Pamela no se sienta junto a Nancy.
    \ii Luz se sienta junto y a la derecha de María y frente a Nancy.
  \end{itemize}
  ¿Dónde se sienta Rita?
\end{mini}
%%a13
\begin{mini}[.9]
  \begin{enum}(2)
    * Junto y entre Nancy y Pamela
    * Frente a Nancy
    * Junto y entre María y Nancy
    * Frente a María
  \end{enum}
\end{mini}
%%r13
\begin{tabular}{c}
  Junto y entre \\
  María y Nancy
\end{tabular}
%%p14.sp
\begin{mini}[.8]
  En un club se encuentran $4$ deportistas cuyos nombres son: Juan, Mario, Luis y Jorge. Los deportes que practican son natación, básquet, fútbol y tenis. Cada uno juega solo un deporte. El nadador que es primo de Juan, es cuñado de Mario y además es el más joven del grupo. Luis que es el de más edad, es vecino del basquetbolista quien a su vez es un mujeriego empedernido. Juan que es sumamente tímido con las mujeres es $7$ años menor que el tenista. ¿Quién practica básquet?
\end{mini}
%%a14
\begin{task}
  * Juan
  * Mario
  * Luis
  * Jorge
\end{task}
%%r14
Mario
%%p15.sp
\begin{mini}[.8]
  Almorzaban juntos $3$ políticos: el señor blanco, el señor rojo y el señor amarillo; uno llevaba corbata blanca, otro corbata roja y otro corbata amarilla pero no necesariamente en ese orden. <<Es curioso>> dijo, el señor de la corbata roja: <<Nuestros apellidos son los mismos que nuestras corbatas, pero ninguno lleva la que corresponde al suyo>>. <<Tiene usted razón>> dijo el señor blanco. ¿De qué color llevaba la corbata el señor amarillo?
\end{mini}
%%a15
\begin{mini}[.6]
  \begin{enum}(2)
    * Blanco
    * Rojo
    * Amarillo
    * Blanco y Rojo
  \end{enum}
\end{mini}
%%r15
Rojo
